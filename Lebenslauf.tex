\documentclass{article}
\usepackage[
a4paper,
margin=0.5cm,
top=1.5cm,
headheight=1cm,
]{geometry}
\usepackage[colorlinks=true]{hyperref}
\usepackage{amsmath}
\usepackage{titlesec}
\usepackage{titling}
\usepackage{fancyhdr}

\pagestyle{fancy}
\fancyhead{}
\fancyhead[R]{\thepage}
\renewcommand{\headrulewidth}{0pt}

%\pagenumbering{gobble}
\renewcommand{\familydefault}{\sfdefault}

\titleformat{\section}
{\huge\bfseries}
{}
{0em}
{}[\titlerule]

\titleformat{\subsection}[runin]
{\bfseries}
{}
{0em}
{}[$\colon$]

\begin{document}

\author{Linkai Zhang}

\section{Linkai Zhang}

\subsection{Geburtsdatum}
05. November 1997

\subsection{E-Mail}
\href{linkai.zhang@xinux.de}{linkai.zhang@xinux.de}

\subsection{Telefon}
06332/44040

\subsection{GitHub}
\href{https://github.com/EinYakAmNil}{https://github.com/EinYakAmNil}

\section{Bildung}

\subsection{Abitur}
Gymnasium Höchstadt 2008-2016

\subsection{Universität}
Integrated Life Science an der Friedrich-Alexander Universität 2016-2020

\subsection{Ausbildung}
Fachinformatik für Systemintegration bei xinux GmbH seit August 2022

\section{Sprachen}

\subsection{gesprochene Sprachen}
Deutsch, Englisch, Chinesisch, Spanisch (B1)

\subsection{Programmiersprachen}
C, Bash, Lua, Python

\subsection{Markup}
\LaTeX, Markdown, HTML, CSS

\section{besondere Kenntnisse}

\subsection{Linux}
Sehr vertraut mit der Funktionsweise und Konfiguration von Linux-Systemen.
(Netzwerk, Firewall, verschiedene Shells und Benutzeroberflächen, Windows-Spiele)

\subsection{Programmieren}
Hauptsächlich Erfahrung mit Webscraping, Datenparsen, -verarbeitung, -darstellung (Python), Skripte schreiben (Bash, Python) und Konfigurieren von Programmen mit Lua-Schnittstellen

\subsection{Laborarbeit}
Laborerfahrung mit verschiedenen chemischen, biologischen (S1 Labor), kristallographischen und physikalischen Experimenten

\subsection{Projektarbeit mit git}
Vertraut mit den grundlegenden Befehlen von git

\subsection{3D-Modellierung}
Erschaffung einfacher 3D-Modelle in Blender

\section{Projekte}

\subsection{CCPBSA}
Code für ein Verfahren zur Berechnung von Energiedifferenzen von Proteinfaltungen in Python

(GitHub: \href{https://github.com/EinYakAmNil/CC-PBSA}{https://github.com/EinYakAmNil/CC-PBSA})

\subsection{iGEM}

\begin{itemize}
\item Datensammlung und -verarbeitung für ein Machine Learning Projekt beim "international Genetically Engineered Machine"-Wettbewerb in 2019
\item Schulung von Studenten des 1. und 2. Semesters im Schreiben von Python-Skripten
\item Projektdauer 1 Jahr
\item Projektwebseite: \href{https://2019.igem.org/Team:FAU_Erlangen}{https://2019.igem.org/Team:FAU\_Erlangen}
\end{itemize}

\section{Schulungen}

\subsection{Schulungsassistent bei:}

\begin{itemize}
\item Linux Grundlagen für hartech Systemhaus GmbH in Dillingen
\item Online: Linux - Security und Firewall für ML Consulting, Schulung, Service \& Support GmbH/KIT/Bundeswehr
\item Linux Härtung für ML Consulting, Schulung, Service \& Support GmbH/Bundeswehr in Köln
\end{itemize}

\subsection{Dozentenerfahrung beim Thema:}

\begin{itemize}
\item Datenschutz/IT-Sicherheit für Stylite AG/xinux GmbH
\item Linux Grundlagen für ML Consulting, Schulung, Service \& Support GmbH/Bundeswehr in Lagerlechfeld
\item Grundlagenwissen für Systemadministratoren für die Bundesakademie für öffentliche Verwaltung (BAköV) in Brühl
\end{itemize}

\end{document}
