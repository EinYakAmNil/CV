\documentclass{article}
\usepackage[
a4paper,
margin=1cm,
top=2cm,
headheight=1cm,
]{geometry}
\usepackage[colorlinks=true]{hyperref}
\usepackage{amsmath}
\usepackage{titlesec}
\usepackage{titling}
\usepackage{fancyhdr}
\usepackage[dvipsnames]{xcolor}
\usepackage{xpatch}

\definecolor{DarkGreen}{RGB}{100, 70, 10}

\pagestyle{fancy}
\fancyhead{}
\fancyhead[R]{\thepage}
\fancyfoot{}
\renewcommand{\headrulewidth}{2pt}
\xpretocmd\headrule{\color{Blue}}{}{\PatchFailed}

\renewcommand{\familydefault}{\sfdefault}

\titleformat{\section}
{\huge\bfseries}
{}
{0em}
{}[{\color{DarkGreen}\titlerule[1.5pt]}]

\titleformat{\subsection}[runin]
{\bfseries}
{}
{0em}
{}[$\colon$]

\titleformat{\subsubsection}[runin]
{}
{}
{0em}
{- }[$\colon$]

\begin{document}

\author{Linkai Zhang}

\section{Linkai Zhang}

\subsection{Geburtsdatum}
05. November 1997

\subsection{E-Mail}
\href{mailto:linkai.zhang@einyamkamnil.xyz}{linkai.zhang@einyakamnil.xyz}

\subsection{Telefon}
+49 15903189871

\subsection{GitHub}
\href{https://github.com/EinYakAmNil}{https://github.com/EinYakAmNil}

\section{Bildung und berufliche Laufbahn}

\subsection{Abitur}
Gymnasium Höchstadt 2008-2016

\subsection{Universität}
Integrated Life Science an der Friedrich-Alexander Universität 2016 - 2020

\subsection{Ausbildung}
Fachinformatik für Systemintegration bei xinux GmbH von 2022 - 2024

\subsection{xinux GmbH}
Fester Mitarbeiter seit Juni 2024

\section{Sprachkenntnisse}

\subsection{gesprochene Sprachen}
Deutsch, Englisch, Chinesisch, Spanisch (B1)

\subsection{Programmiersprachen}
C, Bash, Golang, JavaScript, Lua, Python

\subsection{Markup}
\LaTeX, Markdown, HTML und CSS

\section{besondere Kenntnisse}

\subsection{Linux}
Ich bin LPIC-I zertifiziert (Verifizierung: \href{https://www.lpi.org/verify/LPI000572149/cclmqbtkrb}{https://www.lpi.org/verify/LPI000572149/cclmqbtkrb}).
Ein kleiner Ausschnitt weiterer Erfahrungen beinhaltet:
\begin{itemize}
\item{Mandatory Access Control mit SELinux unter RedHat und Apparmor unter Debian}
\item{Firewalls mit iptables und nftables}
\item{Bash- und Python-Skripting}
\item{verschiedene Tools wie Ansible, Docker, HAProxy und Squid-Proxy}
\end{itemize}

\subsection{Netzwerktechnik}
Ich habe tiefes Verständnis für Netzwerkprotokolle und kenne verschiedene Methoden, um Fehlkonfigurationen zu identifizieren und zu beheben.
Außerdem bin ich sehr vertraut mit der Konfiguration von OPNsense, pfSense und Cisco-Geräten.

\subsection{Monitoring \& Security}
Grundlegende Hacking-Konzepte sind mir bekannt.
In meinen Schulungen habe ich oft Suricata als IDS/IPS vorgestellt und konfiguriert.
Außerdem bin ich vertraut mit Monitoring-Systemen wie Checkmk und dem Elastic-Stack.

\subsection{Virtualisierung}
Ich habe folgende Virtualisierungslösungen bisher benutzt:
\begin{itemize}
	\item Proxmox
	\item VMWare
	\item VirtualBox
	\item Hyper-V
\end{itemize}

\subsection{Programmierung}
Meine bisherigen Programmierprojekte haben folgenden Domänen beinhaltet:
\begin{itemize}
	\item Webscraping
	\item Datenpipelining und -analyse
	\item Entwicklung von Werkzeugen für einfachere Softwareentwicklung
\end{itemize}

\subsection{Laborarbeit}
Ich habe Laborerfahrung mit verschiedenen chemischen, biologischen (S1 Labor), kristallographischen und physikali\-schen Experimenten.

\subsection{3D-Modellierung}
Ich kann mit Modelling-Werkzeugen in Blender umgehen.

\section{Schulungen}

\subsection{Schulungsassistent bei}

\begin{itemize}
\item Linux Grundlagen für hartech Systemhaus GmbH in Dillingen
\item Online: Linux - Security und Firewall für ML Consulting, Schulung, Service \& Support GmbH/KIT/Bundeswehr
\item Linux Härtung für ML Consulting, Schulung, Service \& Support GmbH/Bundeswehr in Köln
\end{itemize}

\subsection{Dozentenerfahrungen}

\begin{itemize}
\item Datenschutz/IT-Sicherheit für Stylite AG/xinux GmbH
\item Linux Grundlagen für ML Consulting, Schulung, Service \& Support GmbH/Bundeswehr in Lagerlechfeld
\item Grundlagenwissen für Systemadministratoren für die Bundesakademie für öffentliche Verwaltung (BAköV) in Brühl
\item Vertiefung Linux/UNIX Netzwerke für die Bundesakademie für öffentliche Verwaltung (BAköV) in Brühl
\item Cyber Schadensabwehr für ML Consulting, Schulung, Service \& Support GmbH/Bundeswehr in Pöcking
\item Backup \& Recovery für ML Consulting, Schulung, Service \& Support GmbH/Bundeswehr in Pöcking
\item Linux - Security und Firewall für ML Consulting, Schulung, Service \& Support GmbH/Bundeswehr in Koblenz und Online
\item Informationssicherheit in Linux-Umgebungen mit Live-Hacking-Demonstrationen für ML Consulting, Schulung, Service \& Support GmbH/Hochschule Meißen (Online)
\item Analyse und Monitoring von Netzwerken für ML Consulting, Schulung, Service \& Support GmbH/Bundeswehr in Dresden
\item Netzwerkdienste und Protokolle als Basis der Optimierung für ML Consulting, Schulung, Service \& Support GmbH/Bundeswehr in Koblenz
\item Netzwerk- und Serveradministration unter Linux für ML Consulting, Schulung, Service \& Support GmbH/KIT/Bundeswehr (Online)
\end{itemize}

\section{Projekte}

\subsection{CCPBSA}
Code für ein Verfahren zur Berechnung von Energiedifferenzen von Proteinfaltungen in Python:

\href{https://github.com/EinYakAmNil/CC-PBSA}{https://github.com/EinYakAmNil/CC-PBSA}

\subsection{iGEM}
\begin{itemize}
\item Datensammlung und -verarbeitung für ein Machine Learning Projekt beim "international Genetically Engineered Machine"-Wettbewerb in 2019
\item Schulung von Studenten des 1. und 2. Semesters im Schreiben von Python-Skripten
\item Projektdauer 1 Jahr
\item Projektwebseite: \href{https://2019.igem.org/Team:FAU_Erlangen}{https://2019.igem.org/Team:FAU\_Erlangen}
\end{itemize}

\subsection{Nvimboat}
Ein erweiterbares RSS-Reader-Plugin für Neovim: \href{https://github.com/EinYakAmNil/Nvimboat.git}{https://github.com/EinYakAmNil/Nvimboat.git}

\subsection{tree-sitter-nftables}
Plattformunabhängiges Syntaxhighlighting für nftables:

\href{https://github.com/EinYakAmNil/tree-sitter-nftables.git}{https://github.com/EinYakAmNil/tree-sitter-nftables.git}

\end{document}
